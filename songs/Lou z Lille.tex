\begin{song}{Lou z Lille}{Klíč}{80pt}
\verse{1}%
Ona [C]jmenuje se [F]Lou a [C]pochází prý z [G]Lille,
má [C]{}čertů rohy, [F]křídla andělů, [C]půvab [G]lesních [C]víl,

kdo strávil s ní pár chvil, jak šampaňské by pil,
má úsměv tupců, trubců závratě pro ni, pro Lou z Lille.

\nchorus{1}%
\rl Jako [C]srpky luny boky [F]tenké má a [C]{}úsměv velkých [G]dam,
tak se [C]lehce [F]vznáší [C]nad ze[G]mí, letí, [C]vůbec [G]netuší, [C]kam.\rr{}

\verse{*}%
[C]Dětsky vážný [F]hlas, větrem [C]urousaný [G]vlas
a [C]oči jako [F]okna za plotem [C]{}černo[G]{}černých [C]{}řas,
co [D]já vím, nemá [G]dům, ale [D]asi ani [A]byt,
a [D]přesto každý [G]kluk chce náramně [D]tam, kde [A]ona, [D]být.

\nchorus{2}%
\rl Jako [D]srpky luny boky [G]tenké má a [D]{}úsměv velkých [A]dam,
tak se [D]lehce [G]vznáší [D]nad ze[A]mí, letí, [D]vůbec [A]netuší, [D]kam.\rr{}

\verse{2}%
Víno, vejce, sýr, taky čerstvých ryb dost má,
tohle do košíku každý na trhu zadarmo jí dá,
a báby závidí, mají každé ráno zlost,
a my si ji tu pěkně hýčkáme jen tak, pro radost.

\nchorus{1} Jako srpky luny boky tenké má\dots

\verse{*}%
Ona [D]jmenuje se [G]Lou a [D]pochází prý z [A]Lille,
má [D]{}čertů rohy, [G]křídla andělů, [D]půvab [A]lesních [D]víl\dots
\end{song}
